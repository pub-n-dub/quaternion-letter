\documentclass[]{article}
\usepackage{todonotes}
\usepackage{xcolor}
\usepackage[colorlinks, citecolor=DeepPink4, linkcolor=DarkRed]{hyperref}
\usepackage{amsfonts, amssymb}
\usepackage{graphicx}
\usepackage{tikz}

\begin{document}

\title{Amp\`ere's Parallel Circuital Currents in Quaternion Notation}

\author{Christopher A. Tucker\\cartheur@pm.me\\}

\date{Submitted: 23 January 2025}

\maketitle

\abstract{Physics is content with understanding of concepts introduced by its earliest practitioners, especially when corresponding mathematics have demonstrated its value. One of such value is the case of action of electric currents, first detailed by Amp\`ere in \textit{Th\'eorie des Ph\'enom\`enes Electrodynamiques}. Although this paper does not pose a challenge to this work, it is instead curious as to its formalism; if additional value in both explicit and intuitive capacities can be offered by writing in Quaternion notation.}

% The introduction.
\section{Introduction}
Amp\`ere's experimental law is stated as follows:

\begin{enumerate}
  \item Equal and opposite currents in the same conductor produce equal and opposite effects on other conductors.
  \item The effect of a conductor bent or twisted in any manner is equivalent to that of a straight one, provided that the two are traversed by equal currents and the former nearly coincides with the latter.
  \item No closed circuit can set into motion an element of a circular conductor about an axis through the center of the circle and perpendicular to its plane.
  \item In similar systems traversed by equal currents, the forces between them are equal.
  \item The action between two elements of currents is a straight line intersecting them.
  \item It is taken for granted that the effect of any element of a current upon another is directly the product of the strengths of the currents and of the lengths of the elements.
\end{enumerate}

% The mathematical proposition of the paper.
\section{Mathematical proposition}
The case is stated thus: Let there be two closed currents whose strengths are $a$ and $a_1$ let $\alpha\prime$, $\alpha_1$ be elements of these, $\alpha$ being the vector connecting their midpoints. Then the effect of $\alpha\prime$ on $\alpha_1$ must, when resolved along $\alpha_1$, be a complete differential with respect to $\alpha$ composed of three independent variables since the total resolved effect of the closed circuit of which $\alpha\prime$ is an element is zero by (3).

The formalism surrounding Quaternions was best espoused by Tait (1890). In elaborating the case, the introduction of tensor $T$, scalar $S$, and unit-tensor $U$ is:

\begin{itemize}
  \item tensor
  \item scalar
  \item unit-tensor
\end{itemize}

% The mathematical proposition as circuital example.
\section{Circuital demonstrative examples}
Words.

\section{Conclusion}
To be written.

\section{Publisher}
\begin{itemize}
  \item  Physics Letters [A](https://www.sciencedirect.com/journal/physics-letters-a)
  \item Article [Submission](https://www.editorialmanager.com/phyla/default.aspx)
\end{itemize}

% Interesting bits.
\section{Errata}
Consider the case of a strand of insulated wire bisected and folded into two wires are crossed at an angle $\theta$. When an electric current, $I$, is applied to the open ends of the wire, the geometry stipulates that the magnetic fields traversing the wire will be dependent with respect to each other's electrical energy, $E$.

At the point of intersection, a toroidal magnetic field rotates and its relative motion is derived from the vector where $v{{B}_{yz}}$, from the intersection containing the largest value of $\theta$ and opposite to that of the vector $v{{B}_{zy}}$, creating a contour, $C$, a contiguous field pattern oscillating on its axis between a zero-valued field and a local maximum at the intersection containing the smallest value of $\theta$. The arrangement is illustrated in Fig.1 with the interior of $C$ shown in Fig.2.

With respect to Fig.2, at the intersection of the wires illustrated in Fig.1, in order to discuss the regions of interest, we draw bisectors of the two angles: one for the smallest angle of $\theta$ and one for the largest angle of $\theta$. We choose an arbitrary radius, $r$, the total region we are interested in performing our analysis, and draw a circle, creating four sub-regions whose centers are scribed with smaller circles, labeling these regions where the model experiences axial torque in quadrants I and III, and a rotational velocity from quadrants II to IV, when current is applied, the model illustrates its magnetic effect upon the space, given the direction of $E_{x}^{a}$ and $E_{x}^{b}$for all times, $t$. Due to the geometry of the arrangement, the direction of rotation of the magnetic field of $E_{x}^{a}$ is opposite to that of $E_{x}^{b}$, yielding region $\alpha$, where the fields emanate outwardly relative to the paper, and region $\beta$, where the fields traverse inwardly relative to the paper, and yielding a toroidal magnetic field rotating from quadrant II to quadrant IV. Locally, $\alpha$ represents maxima of energy due to the stresses [1] in the region, while $\beta$ minima of these forces. At points equidistant from the wires the value of the two fields is the same, however, at locality $\alpha$ the fields double while at locality $\beta$ the fields cancel. Because of the motions and oscillation between the localities, a thin line of zero-valued magnetic field lies in the intersection between quadrants I and III, indicated by the orange region.

Because of the physics in the neighborhood of $\alpha$ and $\beta$, the larger toroidal field, ${{\mathbf{B}}_{w,x,y,z}}$ possesses one-half of its motion against the direction of ${{\vec{B}}^{b}}$ and in favor of ${{\vec{B}}^{a}}$, and the other half of its motion in favor of the direction of ${{\vec{B}}^{a}}$ and against the direction of ${{\vec{B}}^{b}}$. The magnitude of ${{\mathbf{B}}_{w,x,y,z}}$ is directly related to the magnitudes of $E_{x}^{a}$ and $E_{x}^{b}$ respectively.

\end{document}