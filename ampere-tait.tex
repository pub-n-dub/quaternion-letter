\documentclass[]{article}
\usepackage{todonotes}
\usepackage{xcolor}
\usepackage[colorlinks, citecolor=DeepPink4, linkcolor=DarkRed]{hyperref}
\usepackage{amsfonts, amssymb}
\usepackage{graphicx}
\usepackage{tikz}

\begin{document}

\title{Amp\'ere's Circuital Law in Quaternion Notation}

\author{Christopher A. Tucker\\cartheur@pm.me\\}

%\date{Submitted: 30 March 2025\\Revised: xx xxxx 2025}

\maketitle

% The abstract.
\abstract{Physics is content with understanding of concepts introduced by its earliest practitioners, especially when corresponding mathematics have demonstrated its value. One of such value is the case of action of electric currents, first detailed by Amp\`ere in Th\'eorie des Ph\'enom\`enes Electrodynamiques. While this paper does not pose to challenge this work, it is instead curious as to its formalism, and if additional value, both explicit and intuitive can be offered by writing in Quaternion notation.}

% The introduction.
\section{Introduction}
In the case of electrical current in parallel wires, Amp\`ere rigorously proved a steady-state behavior from which all analytics extend. 

% The physical example manifesting the phenomenon.
\section{The circuital model}
Consider the case of a strand of insulated wire bisected and folded into two wires are crossed at an angle $\theta$. When an electric current, $I$, is applied to the open ends of the wire, the geometry stipulates that the magnetic fields traversing the wire will be dependent with respect to each other's electrical energy, $E$.

At the point of intersection, a toroidal magnetic field rotates and its relative motion is derived from the vector where $v{{B}_{yz}}$, from the intersection containing the largest value of $\theta$ and opposite to that of the vector $v{{B}_{zy}}$, creating a contour, $C$, a contiguous field pattern oscillating on its axis between a zero-valued field and a local maximum at the intersection containing the smallest value of $\theta$. The arrangement is illustrated in Fig.1 with the interior of $C$ shown in Fig.2.

With respect to Fig.2, at the intersection of the wires illustrated in Fig.1, in order to discuss the regions of interest, we draw bisectors of the two angles: one for the smallest angle of $\theta$ and one for the largest angle of $\theta$. We choose an arbitrary radius, $r$, the total region we are interested in performing our analysis, and draw a circle, creating four sub-regions whose centers are scribed with smaller circles, labeling these regions where the model experiences axial torque in quadrants I and III, and a rotational velocity from quadrants II to IV, when current is applied, the model illustrates its magnetic effect upon the space, given the direction of $E_{x}^{a}$ and $E_{x}^{b}$for all times, $t$. Due to the geometry of the arrangement, the direction of rotation of the magnetic field of $E_{x}^{a}$ is opposite to that of $E_{x}^{b}$, yielding region $\alpha$, where the fields emanate outwardly relative to the paper, and region $\beta$, where the fields traverse inwardly relative to the paper, and yielding a toroidal magnetic field rotating from quadrant II to quadrant IV. Locally, $\alpha$ represents maxima of energy due to the stresses [1] in the region, while $\beta$ minima of these forces. At points equidistant from the wires the value of the two fields is the same, however, at locality $\alpha$ the fields double while at locality $\beta$ the fields cancel. Because of the motions and oscillation between the localities, a thin line of zero-valued magnetic field lies in the intersection between quadrants I and III, indicated by the orange region.

Because of the physics in the neighborhood of $\alpha$ and $\beta$, the larger toroidal field, ${{\mathbf{B}}_{w,x,y,z}}$ possesses one-half of its motion against the direction of ${{\vec{B}}^{b}}$ and in favor of ${{\vec{B}}^{a}}$, and the other half of its motion in favor of the direction of ${{\vec{B}}^{a}}$ and against the direction of ${{\vec{B}}^{b}}$. The magnitude of ${{\mathbf{B}}_{w,x,y,z}}$ is directly related to the magnitudes of $E_{x}^{a}$ and $E_{x}^{b}$ respectively.

% The mathematical proposition to explain the observed phenomenon.
\section{The mathematical hypothesis}
At the intersection, we have two wires that interact magnetically only, since there is no electrical connection between each. We create a cut in the form of a spherical measurement space, for convenience and to denote the boundary conditions we are most interested and within the influence of the toroidal-shaped magnetic field-object (a contiguous object manifesting a continuous contour), $C$, whose center is shared with the intersection of the two wires. To perform a measurement, we report at it at a specific one-dimensional space, as a point, and in a two dimensional space, such as a stripe. A stripe is a virtual object that acts as a measurement probe where it theoretically has no influence upon the field by its presence.

For the arrangement illustrated in Fig.2, in the case of a non-sinusoidal current, ${{I}_{0}}$, consider the partial denotation of the composite behavior, a local examination of a suspected nullity is illustrated by Figure 3.

In the first instance it must be stated that there are no currently supported methods to map the entirety of a magnetic field, and its dynamics, other than extrapolation by “force lines”, originally devised by Faraday. However, we are interested in force in our model, but are more curious as to happenings in the planar sheet forming a contour at arbitrarily-selected regions in the occupied space, we denote as a field object. Therefore, we are curious as to the rotations of different areas of this object because we assume that behavior by dynamics and field intensity in a given region is due to a corresponding event elsewhere in the object. Also, it is interesting what kind of coupled-mode is going on between those regions showing demonstratively paradoxical rotations. Furthermore, the intersection of these regions is fuzzy and can only be described by rotations (quaternions) and force-lines (magnitudes). Although each of these quantities are discrete separately, they are conjoined in a higher-order  mathematical and potentially real-valued space, at imaginary coordinates relative to the quantized model of the electron (and matter generally) described in the Standard Model traversing the conductor at a given intensity $E$. It is within the higher-order space that coupling between distant features in the object is located, resulting in a real-factored link-potential between regions. Potentially, coupled-modes can exist in an infinite number for the case of inductors, we argue that it is within this transmathatical space which facilitates a reasonable mathematical description of behavior, parameters, and characteristics of the different regions, relative to the potential linking of the two rotating and relatively-oscillating field regions, illustrated in Fig.2.

The dynamical behavior illustrated in Fig.4 is global, that is, it is what is manifest as a result of a current being passed through the wire. But as Fig.4 illustrates, there are three distinct motions of the magnetic field, $B$, resulting from the electrical energy, $E$, passing through the wire. In order of trigger, $B_{yz}$, $B_{rot}$, and $B_{wxyz}$. $B_{yz}$ is driving the motion of the subsequent two.

When examining the physical manifest properties of the wire, measuring the details of its dynamics to verify the hypothesis presented here is key [2], although its dynamics can be predicted based on simple representational modeling [3]. However, for the purposes of this paper, we want to understand the interaction between each area manifesting different dynamical behavior. We will then want to understand what kind of mathematical operation is when examining the behavior across each area. In order to accomplish this, we here introduce the concept of the stripe.

% Discussion of results.
\section{Discussion of results}
This section may be unnecessary.

\section{Conclusion}
To be written.

\end{document}